%%%%%%%%%%%%%%%%%%%%%%%%%%%%%%%%%%%%%%%%%
% Short Sectioned Assignment
% LaTeX Template
% Version 1.0 (5/5/12)
%
% This template has been downloaded from:
% http://www.LaTeXTemplates.com
%
% Original author:
% Frits Wenneker (http://www.howtotex.com)
%
% License:
% CC BY-NC-SA 3.0 (http://creativecommons.org/licenses/by-nc-sa/3.0/)
%
%%%%%%%%%%%%%%%%%%%%%%%%%%%%%%%%%%%%%%%%%

%----------------------------------------------------------------------------------------
%	PACKAGES AND OTHER DOCUMENT CONFIGURATIONS
%----------------------------------------------------------------------------------------

\documentclass[paper=a4, fontsize=11pt]{scrartcl} % A4 paper and 11pt font size

\usepackage[T1]{fontenc} % Use 8-bit encoding that has 256 glyphs
\usepackage{fourier} % Use the Adobe Utopia font for the document - comment this line to return to the LaTeX default
\usepackage[english]{babel} % English language/hyphenation
\usepackage{amsmath,amsfonts,amsthm} % Math packages
\usepackage{hyperref}
\hypersetup{colorlinks=true,
	linkcolor=blue,
	filecolor=magenta,      
	urlcolor=blue,}
\usepackage{sectsty} % Allows customizing section commands
\allsectionsfont{\centering \normalfont\scshape} % Make all sections centered, the default font and small caps

\usepackage{fancyhdr} % Custom headers and footers
\pagestyle{fancyplain} % Makes all pages in the document conform to the custom headers and footers
\fancyhead{} % No page header - if you want one, create it in the same way as the footers below
\fancyfoot[L]{} % Empty left footer
\fancyfoot[C]{} % Empty center footer
\fancyfoot[R]{\thepage} % Page numbering for right footer
\renewcommand{\headrulewidth}{0pt} % Remove header underlines
\renewcommand{\footrulewidth}{0pt} % Remove footer underlines
\setlength{\headheight}{13.6pt} % Customize the height of the header

\numberwithin{equation}{section} % Number equations within sections (i.e. 1.1, 1.2, 2.1, 2.2 instead of 1, 2, 3, 4)
\numberwithin{figure}{section} % Number figures within sections (i.e. 1.1, 1.2, 2.1, 2.2 instead of 1, 2, 3, 4)
\numberwithin{table}{section} % Number tables within sections (i.e. 1.1, 1.2, 2.1, 2.2 instead of 1, 2, 3, 4)

\setlength\parindent{0pt} % Removes all indentation from paragraphs - comment this line for an assignment with lots of text

%----------------------------------------------------------------------------------------
%	TITLE SECTION
%----------------------------------------------------------------------------------------

\newcommand{\horrule}[1]{\rule{\linewidth}{#1}} % Create horizontal rule command with 1 argument of height

\title{	
\normalfont \normalsize 
\textsc{CMPS 263} \\ [25pt] % Your university, school and/or department name(s)
\horrule{0.5pt} \\[0.4cm] % Thin top horizontal rule
\huge Data Wrangling Procedure \\ % The assignment title
\horrule{2pt} \\[0.5cm] % Thick bottom horizontal rule
}

\author{Brady Goldman, James Byron, Jon Selberg} % Your name

\date{\normalsize\today} % Today's date or a custom date

\begin{document}

\maketitle % Print the title

\section{Data Retrieval}

\begin{itemize}
	\item \textbf{System Requirements}
	\begin{itemize}
		\item Java SE Runtime Environment 8
		\item Internet Explorer (Not Edge) or Mozilla Firefox
	\end{itemize}
	\item \textbf{Data Retrieval from ACS using DataFerrett}
	\begin{itemize}
			\item Before running DataFerrett, be sure to disable your popup blocker for  \url{https://dataferrett.census.gov} and allow Java to run in your browser.
			\item Go to \url{https://dataferrett.census.gov/LaunchBetaDFA.html} and enter your ucsc email in the login popup.  Then click the top tab that says \textbf{Step1}.
			\item From the left column, dropdown into the American Community Survey dataset, then dropdown into the Public Use Microdata Sample.  View variables from the 2015 dataset.
			\item Only select the topic that says \textbf{Population}, then search the variables.
			\item Double click variables with names DEAR, DEYE, DPHY, OCCP, SCHL, and WAGP.  Select all values for each varable with the exception of "Not in universe - missing" for DPHY.
			\item Proceed to \textbf{Step2} tab.  Click button that says \textbf{Make A Table}.
			\item Drag and drop OCCP variable to left-most column.  Then drag DEAR and WAGP in that order to the adjacent columns and Click the green button that says \textbf{GO Get Data}.  When the data has replaced the question mark placeholders, save the table as a .csv file.  Clear spreadsheet and repeat for DEYE and DPHY.
			\item Drag and drop the OCCP variable in the left-most column again, but this time drag only SCHL to the column adjacent.  Generate data just like above and save to a file titled "education\_attained\_by\_occupation.csv".
			\item Place each csv file in a folder labeled "data".
	\end{itemize}
\end{itemize}

\section{Data Post-Processing}

\begin{itemize}
	\item \textbf{System Requirements}
	\begin{itemize}
		\item Python 2.7.x with numpy and pandas installed
		\item Excel or Libre office
	\end{itemize}
	\item \textbf{Narrowing Education Categories}
	\begin{itemize}
		\item Open file "education\_attained\_by\_occupation.csv" in Libre office or Excel.  This should have counts for each highest level of education attained by occupation.  Group these into the following five categories: Less Than High School; High School; Some College; Bachelor Degree; and Master, PhD, or Professional Degree.
		\item Sum counts from Columns D-S to obtain total by occupation for "Less than High School", Columns T-U for "High School", V-X for "Some College", Y for "Bachelors", and Z-AB for "Master, PhD, or Professional Degree".  Place these in the next five columns after the last.  
	\end{itemize}
	\item \textbf{Mapping expected highest level of education attained by occupation}
	\begin{itemize}
		\item To convert the counts for highest level of education attained to a label for each occupation, we compute the following expected value for each:
		\begin{align*}
		\frac{\sum_{i=0}^{4}{i*\text{Count}_i}}{\text{Total Count}}
		\end{align*}
		Where $i$ is the index corresponding to each category of highest degree obtained.  The result rounded to the nearest whole number will designate the corresponding category as the label.
		\item Download and move expected\_education\_mapper.py just outside of the directory labeled "data".  Run script to generate csv file with categories mapped to each occupation.
	\end{itemize}
\end{itemize}

\end{document}